\documentclass{article}
\usepackage{url}
\usepackage{hyperref}
\usepackage{fullpage}
\usepackage{color}

\newcommand{\tl} {\textless}
\newcommand{\tg} {\textgreater}
\renewcommand{\labelitemi}{$\vcenter{\hbox{\tiny$\bullet$}}$}

\definecolor{darkgreen}{rgb}{0,0.5,0}
\definecolor{darkblue}{rgb}{0,0,0.5}
\hypersetup{colorlinks,
linkcolor=darkgreen,
urlcolor=darkblue}
\begin{document}
\noindent{\Large How to convert a latex document into a Word .docx}\\
Author: Laura Tremblay-Boyer, l.boyer@fisheries.ubc.ca\\
\\
\emph{Note 1: This was written for Mac/UNIX users, but most instructions should be
valid for Windows.}\\
\\
\emph{Note 2: There are many ways to do this, and I have found none at this stage
that are perfect. The best option in my opinion is to use latex2rtf
(see section \ref{sec::rtf}) which converts .tex to a .rtf (Rich Text File),
which is in turn readable by Word. The manual changes that you have to
do for the latter are minimal, but it is a tad confusing to
install. Once that hurdle is past though, you are good to go. I
included instructions for another option, Pandoc (see section \ref{sec::pandoc}), which is very easy
to install and can handle conversions between many, many file formats -- but
the resulting .docx lacks some features that are a pain to correct
manually (e.g. in-text references). Still, it is a good ressource to
have given its versatility.}\\
\\
\emph{Both latex2rtf and Pandoc requires you to use the command line,
  via the Terminal (which apparently has an equivalent in Windows called
PowerShell? ...unverified). To find it, type ``Terminal'' in the
Spotlight search bar.}\\
\\
\noindent $\rightarrow$ to view the instructions for \emph{latex2rtf}, \hyperref[sec::rtf]{click here}.\\
\noindent $\rightarrow$ to view the instructions for \emph{Pandoc}, \hyperref[sec::pandoc]{click here}.

\section{The almost-quick-and-easy-and-almost-perfect way:
  latex2rtf}
\label{sec::rtf}
\emph{I wrote this for Mac/UNIX users, see this webpage for instructions
  for Windows:} \url{http://latex2rtf.sourceforge.net/usage.html}.

\subsection{Installation}
\begin{enumerate}
\item Download latex2rtf and unpack it:
  \url{http://sourceforge.net/projects/latex2rtf/}
\item Open the terminal and navigate to the unpacked \texttt{latex2rtf}
  directory, for e.g.: \\
\texttt{cd ~/Downloads/latex2rtf-2.3.3/}
\item Once in the directory, type the following commands one at a
  time:\\
\\
\texttt{make}\\
\\
\texttt{make check}\\
\\
\texttt{make install}\\
\\
If you get a 'Permission denied...' message here, try adding $sudo$ before the
command (it will ask for the administrator's password, and should then
proceed normally):\\
\texttt{sudo make install}
\item Verify that latex2rtf installed properly by typing: \\
\texttt{latex2rtf -v}\\
... if this produces something intelligible, you are good to go.
\item If this didn't work for you, darn. You probably have to make
  changes to the Makefile. Open in a text-editor the README that comes
  with the program for further instructions. If you are using a Mac
  you might get a message that this comes from an unidentified
  developer and cannot be opened. CTRL-click Open the file
  name and it should let you open the file despite the warning.
\end{enumerate}

\subsection{Converting .tex to .RTFs}

\begin{enumerate}
\item Open the Terminal and go to the directory containing the latex file
  you want to convert:\\
\texttt{cd ~/Documents/.../my-directory-of-choice}
\item (Optional) Once in the directory, you can use \texttt{ls} to list
  the files it contains --- you should be able to see the file
  you want to convert.
\item Create a copy of your file (just in case, but probably not necessary).
\item Type the following command:\\
\texttt{latex2rtf your-file-name.tex} \\
... this should produce the file \texttt{your-file-name.rtf}. Note the
.rtf extension. During the conversion, watch out for warnings of the type ``Package/option ... unknown'' --
you will have to modify manually whatever settings were changed by
these options as they were not imported in the .rtf.
\item Open the \texttt{your-file-name.rtf} in Word... Most things should be
  there, but have a scan through to see if anything is missing. For
  instance if you used the package \texttt{fullpage} you will have to
  change the margins of the Word document back to 2.5cm. If you had
  pages in landscape format you'll have to add
  those manually by adding a section break before the desired
  landscape page and changing the orientation for that page only.
\item When you are done use Word to save the .rtf to a .docx. And you
  are done. Party!
\end{enumerate}




\section{The kinda-quick-and-easy-but-not-perfect way: Pandoc}
\label{sec::pandoc}
This method handles 95\% of formatting, figures and bibliography --- and
it comes with its own installer (just
double-click the .dmg). It doesn't do mixed portrait and landscape pages, in-text references to
figures and tables, and probably other stuff that I can't think of
right now. However it can convert between a number of file formats,
see \url{http://johnmacfarlane.net/pandoc/diagram.png} (... \emph{right?!}).

Once you have specified the right options, the output .docx doesn't require manual alterations. It is probably
good for sending to a supervisor or collaborator (who you will have forewarned of
minor formatting issues) but not as the final version to, say, a thesis
committee or a journal.\\
\\
Here goes:

\begin{enumerate}
\item Download and install Pandoc: \url{http://johnmacfarlane.net/pandoc/installing.html}
\item Open the Terminal and confirm that
  Pandoc is installed by typing:\\
\texttt{pandoc {--}version}\\ (if it is not installed it will print \texttt{command not found}).
\item Navigate to the directory containing the file you want to
  convert using the command \texttt{cd} (use \texttt{pwd} to see where you are
  presently):\\
\texttt{cd ~/Documents/Manuscripts/}
\item (Optional) Once in the right directory, you can use \texttt{ls} to list
  the files in the directory --- you should be able to see the file
  you want to convert.
\item Create a copy of your file (just in case).
\item Now, tell pandoc to convert your file from latex to docx:\\
\texttt{pandoc your-file-name.tex -f latex -t docx -s -o
 output-file-name-of-your-choice.docx \tl other options as
needed \tg}
\begin{itemize}
\item -f \tl...\tg: the format you are converting \emph{from}
\item -t \tl...\tg: the format you are converting \emph{to}
\item -o \tl...\tg: the desired \emph{output} file name, with the
  extension
\end{itemize}
\item To see the many options that can be added to the pandoc command,
  go here: \url{http://johnmacfarlane.net/pandoc/README.html#options}.
\item Some miscellaneous things that should come in handy:

\begin{itemize}
\item Bibliography: If you have a bibliography, tell Pandoc where to find it by
  adding the option \texttt{ {--}bibliography = your-bibliography-file-name.bib}.

\item Figures: At this stage your document looks pretty good \emph{except} for wonky
  figure sizes which do not match those you defined in your .tex
  (shoot). Here's why: Pandoc imports the figure at its
  original size and ignores the size settings you defined in the
  .tex. Thanks Pandoc. The easy way around this (and this is only a
  pain once) is to make sure that you always produce your figures in a
  size that is adequate for a Word document. This way you can fiddle
  with size in the .tex, but when you convert it to Word the figure
  looks good. If you produce directly the figure from R you can define
  a width and height in your code --- e.g. pdf(..., width = X, height=
  Y). If you have a Mac you can open the image in Preview, go to Tools
  $\rightarrow$ Adjust size and modify the size directly. As a rule of
  thumb I find that using the width of the between-margins distance of
  a standard Word document works well --- in my case that is 15.8cm,
  or 6.22 inches. If you get in the habit of producing images of an
  appropriate size as you go, this is not too much of a pain. And this
  way you can also set the DPI to have a figure resolution that makes
  you happy.

\item Error messages: Could not find image '...': go back to your .tex
  and make sure the extension is included in the image's file name, or
  if all your images have the same extension add the option
  \texttt{{--}default-image-extension=...}.

\item In-text references: Pandoc does not translate in-text references to Word (which is
  too bad as that is one of the coolest features of Latex)(that being
  said Word usually doesn't handle those well --- but that's another
  story). What Pandoc does instead is copy the value of the label/ref
  literally in the Word document. For instance if you have in .tex
  ``see Figure {\textbackslash ref\{fig::dispersal\}}'', which would render in the PDF
  as ``see Figure 4'' (and the 4 would be clickable), in the Pandoc translation .docx it will look
  like ``see Figure [fig::dispersal]''. Adding option \texttt{{--}parse-raw} removes
  the [...] labels as specified in the .tex's \textbackslash label and
  \textbackslash ref, which looks cleaner but gives no indication to
  your reader of the figure, table or section you are linking to. My
  suggestion is to leave the [...] labels in (i.e. not use the
  \texttt{{--}parse-raw} option) and warn the reader that they can search for
  the label using CTRL-F to find the figure, table or section that is
  being referred to.

\end{itemize}

\end{enumerate}
\end{document}